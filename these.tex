\documentclass[12pt,a4paper]{report}
\usepackage[T1]{fontenc}
\usepackage[utf8]{inputenc}
\usepackage[francais]{babel}

\usepackage{geometry}
\geometry{hmargin=3cm,vmargin=3cm}
\linespread{1.2}

\begin{document}
\chapter{La mission Athena}
Dans l'univers il existe ce que l'on appelle des phénomènes d’accrétion, il s'agit du phénomène d'agglomération de matières. On peut observer ces phénomènes à des échelles très différentes allant des disques proto-planétaires jusqu'aux noyaux actifs de galaxie. En astronomie, la majorité des sources sont détectées dans le spectre de rayonnement X et les phénomènes d'accrétions en sont à l'origine. Les traces de ceux-ci sont observables autour de trous noirs ou d’étoiles à neutrons dont la densité est très élevées et dépasse même celle de la matière nucléaire. Les disques d’accrétion se formant autour de ces objets atteignent des température de plusieurs millions de degré. L’émission X qui en résulte provient des parties les plus internes de ces systèmes. Beaucoup d'objets compacts ont été détectés grâce à l'accrétion de matières, l'étude de ces rayons X peut donc nous en apprendre beaucoup sur l'univers  plus précisément sur la formation de structures.

La mission Advanced Telescope for High ENergy Astrophysics (Athena) a été sélectionné en 2014 (Nandra et al.,2013) pour approfondir le thème de "l'univers chaud et énergétique", c'est la seconde mission la plus importante de l'Agence spatiale européenne. Le télescope avancé pour l'astrophysique des hautes énergies sera la deuxième grande classe mission du programme spatial scientifique Cosmic Vision de l’Agence spatiale européenne (ESA). Cette prochaine génération de télescope à rayons X de génération abordera une variété de questions scientifiques clés allant de la formation l’évolution des groupes et des amas de galaxies, l’histoire de l’enrichissement chimique de l’univers et la baryons manquants, à la formation des premiers trous noirs supermassifs, leur rôle dans l’évolution des les galaxies et la physique de l'accrétion (Pointecouteau et al., 2013; Ettori et al., 2013; Croston et al.,2013; Kaastra et al., 2013; Aird et al., 2013; Georgakakis et al., 2013; Cappi et al., 2013; Dovciak et al., 2013). Conçu comme un observatoire à rayons X ouvert et générique, il permettra également une avancée décisive. Ccapacités pour un large panel de sujets astrophysiques. Certains ont été historiquement étudiés dans les rayons X comme des objets compacts, des supernovae ou des étoiles massives, mais d’autres comme des exoplanètes peuvent bénéficier d’un nouveau fenêtre dans leur domaine (Branduardi-Raymont et al., 2013; Sciortino et al., 2013; Motch et al., 2013; Decourchelle et al., 2013; Jonker et al., 2013). Le lancement du satellite Athena sur une Ariane VI en 2028 est prévu pour l’une ou l’autre des deux premiers points lagrangien Soleil-Terre (L1 / L2) à avoir en même temps un thermique très stable environnement ainsi qu’une efficacité d’observation optimale et une bonne visibilité du ciel. Il effectuera pointu des observations allant de 1 ks à 1M mais réagissent également rapidement (moins de 4 heures) aux Alertes de cible d'opportunité (TOO) afin d'observer des sources transitoires telles que des sursauts gamma (GRB) rémanence ou supernovae.

La mission Athena embarquera un télescope à rayons X et deux instruments dont un spectromètre intégral de champ (le X-IFU pour X-ray Integral Field Unit) et un imageur large champ (le WFI pour Wide Field Imager). Dans cette thèse nous nous intéresserons tout particulièrement au X-IFU, cet instrument ainsi que toute la chaîne de traitement de données au sol sont développés par un consortium scientique international sous responsabilité française. Le CNES assure la maîtrise d'oeuvre de l'instrument et l'IRAP assure la responsabilité scientifique mais d'autres laboratoires français interviennent aussi dans la conceptionde l'instrument tels que le CEA ou encore l'APC.
\end{document} 
